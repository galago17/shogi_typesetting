\documentclass{article}

\usepackage{graphicx}
\usepackage{multicol}
\usepackage{anyfontsize}
\usepackage{forloop}
\usepackage{comment}

\graphicspath{ {./img/} }

\setlength\columnsep{1in}
\setlength{\tabcolsep}{0.5in}


% Medium Sized Board Diagram
\newcommand{\imgmed}[2]{
\begin{center}
	\Large{\textbf{#1}}
	\newline
	\begin{minipage}{\textwidth}
		\centering
		\includegraphics[scale=0.4]{#2}
	\end{minipage}
	\centering
\end{center}
}

% Small Sized Board Diagram
\newcommand{\imgsmall}[2]{
	\begin{minipage}{\textwidth}
		\hspace{1.25cm}\Large{\textbf{#1}}
		\newline
		\includegraphics[scale=0.28]{#2}
	\end{minipage}
}

% Side by Side (2 column) Commentary
\newcommand{\sidebyside}[2]{
\begin{minipage}{\textwidth}
\begin{multicols*}{2}
\begin{minipage}{0.5\textwidth}
#1
\end{minipage}
\columnbreak
\nopagebreak
\begin{minipage}{0.5\textwidth}
	#2
\end{minipage}
\end{multicols*}
\end{minipage}
}

% Title Size, centered, for captions
\newcommand{\bfcenter}[1]{
\centering
\Large{\textbf{#1}}
}

% Unindented Paragraph with space afterwards, for use in Side by Side Commentary
\newcommand{\colpara}[1]{
\noindent#1\bigskip
}

\newcommand{\fullpage}[2]{
	\begin{center}
	{\fontsize{50}{60}\selectfont \textbf{#1}}
	\vspace{1cm}
	\includegraphics[scale=0.7]{#2}
	\end{center}
}

% Tabular Moves
\newcounter{movenumber}
\newcommand{\movenumber}{\stepcounter{movenumber}\arabic{movenumber}}
\newcommand{\cm}[2]{\movenumber & #1 & #2 \\}

\newcommand{\sm}[1]{\movenumber & #1\\}



% Annotated Game Env
\newenvironment{chessmoves}[1]
{
\Large
\setcounter{movenumber}{#1}
\begin{tabular}{l l l}
}
{ 
\end{tabular} 
}

\newenvironment{shogimoves}[1]
{
\Large
\setcounter{movenumber}{#1}
\begin{tabular}{l l}
}
{ 
\end{tabular} 
}





\begin{document}
\large
\title{\textbf{gosttone vs. Tuelyp}}
\author{\textit{by LilyLionMane, typeset by Galago}}
\date{(Gosttone - Tuelyp; TTM10, 81Dojo, 2024)}
\maketitle


\imgmed{``The Attack Rolls in Like a Sudden Tide''}{diagram1}

Though I elected not to play in this season (season 10) of the TT-Series, I did so with an interesting intent. In fact, I wanted to focus on writing about the games and making a column on them. This is the first game I chose. It's from the first round of the tournament in the Masters' section (just below the top section) ``TTM''.

The players are both individuals who I've known for quite some time. Gosttone (2-Dan), Sente in this game, plays the second board on Guatemala's World Shogi League team.

Tuelyp (1-Dan) is also a very strong player, especially as of the past few weeks, and is a Twitch streamer. I think she should be promoted to 2-Dan in the coming days.
\pagebreak

\sidebyside
% First Column
{
\imgsmall{After 11. P-25}{diagram2}
\newline
\imgsmall{After 21. N-37}{diagram3}
}
% Second Column
{
\normalsize
\colpara{As the critical point in this game (pictured above) is from Gote's perspective, we'll look from her position throughout the game.}

\texttt{ 
\colpara{1.P-76 2.P-54 3.P-66 4.R-52 5.S-78 6.P-55 7.S-67 8.P-34 9. P-26 10.S-42 11.P-25}}

\colpara{Sente's plans will be affected by the closed bishop diagonal.}

\texttt{\colpara{12.B-33 13.S-48 14.S-53 15.P-46 16.K-62 17.S-47 18.K-72 19.P-36 20.S-54 21.N-37}}

\colpara{Gote is somewhat faster to castle, almost finished with Half-Mino or else with some kind of 3-Move Castle. Sente's shape is very balanced and could become many different things.}

\noindent Gote has many options to continue from this position. Most natural may be to continue castling.
}

\pagebreak

\begin{chessmoves}{}
	\cm{e4}{e5}
	\cm{Nf3}{Nf6}
	\cm{Bb5}{a6}
\end{chessmoves}

\vspace{1in}
\hrule
\vspace{1in}

\begin{shogimoves}
	\sm{G'22}
\end{shogimoves}

\end{document}